\documentclass[aspectratio=169]{beamer}

\usepackage{../theme/polygl0ts}
\usepackage{drawstack}

\tikzstyle{freecell}=[fill=none]

\newcommand{\reg}[1]{\mintinline{asm}{#1}}
\newcommand{\hex}[1]{\mintinline{python}{0x#1}}
\newcommand{\naddr}[2]{\begin{tabular}{l}#1\\\hex{#2}\end{tabular}}
\newcommand{\docl}[1]{(\textbf{\href{#1}{Documentation}})}

\setbeamertemplate{navigation symbols}{}
% \setbeameroption{show notes on second screen=left}

%%%%%%%%%%%%%%%%%%%%%%%%%%%%%%%%%%%%%%%%%%%%%%%%%%%%%%%%%%%%%%%%%%%%%%%%%%%%%%%
% Title Setup
%%%%%%%%%%%%%%%%%%%%%%%%%%%%%%%%%%%%%%%%%%%%%%%%%%%%%%%%%%%%%%%%%%%%%%%%%%%%%%%
\title{Introduction to Reverse Engineering}
\subtitle{or: how to make everything open source}
\author{therokdaba}
\date{\today}

\begin{document}
\titleframe

% \tocframe

\begin{frame}
    \frametitle{What's rev?}
    
    \begin{itemize}
        % \note{note text} % TODO fix notes
        \pause
        \item Stands for reverse engineering
        \pause
        \item You have a binary, no source code and you want to do understand what it does
        \pause
        \item For example:
            \begin{itemize} % TODO maybe improve the examples
                \pause
                \item Bypassing license checks, DRM, etc.
                \pause
                \item Vulnerability research
                \pause
                \item Malware analysis
                \pause
                \item ... and CTFs!
            \end{itemize}
    \end{itemize}
\end{frame}

\begin{frame}
    \frametitle{What are rev challenges?} 
    
    \begin{itemize}
        \item You get a binary (sometimes stripped, no symbols)
        \pause
        \item It contains the flag (or a way to get it)
        \pause
        \begin{itemize}
            \item Sometimes it checks if the input is the flag
            \pause
            \item Sometimes it encrypts the flag 
            \pause
            \item Sometimes you need to solve something else to get the flag % TODO rephrase 
        \end{itemize}
        \pause
        \item Understand the logic of the binary to get the flag
    \end{itemize}
\end{frame}

\begin{frame}
    \frametitle{What are rev techniques?}
    \begin{itemize}
        \pause
        \item Static analysis (looking at the binary without running it)
        \pause
        \item Dynamic analysis (running the binary and observing its behavior, uses a debugger and things like code patching)
        \pause
        \item Best: both!
    \end{itemize}
\end{frame}

\begin{frame}[fragile]
    \frametitle{What's a binary?}
    \pause
    \texttt{Binary files are the compiled version of source code}\\
    \pause
    More precisely:
    \begin{codebox}{objdump-nasm}
\$ file /bin/ls
/bin/ls: ELF 64-bit LSB pie executable, x86-64, version 1 (SYSV), 
dynamically linked, interpreter /lib64/ld-linux-x86-64.so.2, 
BuildID[sha1]=baba64ae8c7edfd5d7c2b72b425d45ab72a46785, 
for GNU/Linux 4.4.0, stripped
    \end{codebox}
\pause
    \begin{codebox}{objdump-nasm}
\$ hexdump -C /bin/ls | head -n 5
00000000  7f 45 4c 46 02 01 01 00  00 00 00 00 00 00 00 00  |.ELF............|
00000010  03 00 3e 00 01 00 00 00  30 51 00 00 00 00 00 00  |..>.....0Q......|
00000020  40 00 00 00 00 00 00 00  c0 13 02 00 00 00 00 00  |@...............|
00000030  00 00 00 00 40 00 38 00  0d 00 40 00 1c 00 1b 00  |....@.8...@.....|
00000040  06 00 00 00 04 00 00 00  40 00 00 00 00 00 00 00  |........@.......|
    \end{codebox}
\end{frame}

\begin{frame}
    \frametitle{What's an ELF?}
    \pause
    \texttt{UNIX file format}
    \begin{itemize}
        \pause
        \item entry point
        \pause
        \item segments (map ELF file → RAM)
        \pause
        \item sections (map segments to sections)
        \pause
        \item dynamic linking (load more ELFs)
        \begin{itemize}
            \item libc in particular!
        \end{itemize}
    \end{itemize}
\end{frame}

\begin{frame}[fragile]
    \frametitle{Where does it start?}
    \texttt{\$ readelf -h /bin/ls}
    \begin{codebox}{objdump-nasm}
ELF Header:
  Magic:   7f 45 4c 46 02 01 01 00 00 00 00 00 00 00 00 00 
  Class:                             ELF64
  Data:                              2's complement, little endian
  Version:                           1 (current)
  OS/ABI:                            UNIX - System V
  ABI Version:                       0
  Type:                              DYN (Position-Independent Executable file)
  Machine:                           Advanced Micro Devices X86-64
  Version:                           0x1
  Entry point address:               0x5130
  Start of program headers:          64 (bytes into file)
  Start of section headers:          136128 (bytes into file)
  Flags:                             0x0
  Size of this header:               64 (bytes)
  ...
    \end{codebox}
\end{frame}

\begin{frame}[fragile]
    \frametitle{What runs?}
    \begin{codebox}{asm}
\$ objdump -Mintel -d /bin/ls
endbr64
xor    ebp,ebp
mov    r9,rdx
pop    rsi
mov    rdx,rsp
and    rsp,0xfffffffffffffff0
push   rax
push   rsp
xor    r8d,r8d
xor    ecx,ecx
lea    rdi,[rip+0xffffffffffffdf31]
call   QWORD PTR [rip+0x1cb3b] # 21c90 <__libc_start_main@GLIBC_2.34>
    \end{codebox}
\end{frame}

\begin{frame}[fragile]
    \frametitle{Tools}
        \begin{itemize}
            \pause
            \item \texttt{file} to figure out what kind of binary it is
            \pause
            \item \texttt{strings} to get printable strings from the binary
            \pause
            \item \texttt{ltrace}/\texttt{strace} to trace library/syscalls
            \pause
            \item \texttt{gdb} for dynamic analysis
            \pause
            \begin{itemize}
                \item \href{https://github.com/hugsy/gef}{GEF}
                \item \href{https://pwndbg.re/}{pwndbg}
                \item \href{https://github.com/longld/peda}{PEDA}
            \end{itemize}
            \pause
            \item \texttt{objdump}/\texttt{readelf} for static analysis
            \pause
            \item IDA Pro, Ghidra, radare2, Binary Ninja for disassembly/decompilation
            \pause
            \item \texttt{pwntools} for exploit development
        \end{itemize}
\end{frame}

\begin{frame}[fragile]
    \frametitle{decompilation}
    \texttt{go from disassembly to "source code"} 
    \pause
    \begin{codebox}{c}
void processEntry entry(undefined8 param_1,undefined8 param_2){
  undefined auStack_8 [8];
  
  __libc_start_main(FUN_20,param_2,&stack8,0,0,param_1,auStack_8);
  do {
    /* WARNING: Do nothing block with infinite loop */
  } while( true );
}
    \end{codebox}
\end{frame}

\begin{frame}
    \frametitle{Ghidra}
    \begin{itemize}
        \item \href{https://ghidra-sre.org/}{https://ghidra-sre.org/}
        \item Developed by the NSA.open source, free
        \item There are other tools like 
        \begin{itemize}
            \item \href{https://hex-rays.com/ida-free}{IDA (Free)}
            \item \href{https://binary.ninja/}{Binary Ninja}
            \item \href{https://github.com/radareorg/radare2}{radare2} 
            \item \href{https://cutter.re/cutter-2.0}{Cutter (GUI for radare2)}
        \end{itemize}
    \end{itemize}
\end{frame}

\end{document}